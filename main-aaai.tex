%File: formatting-instruction.tex
\documentclass[letterpaper]{article}
\usepackage{aaai}
\usepackage{times}
\usepackage{helvet}
\usepackage{courier}
\usepackage{graphicx}
\usepackage{url}

\usepackage{amsmath, amssymb}
% %%%%%%%%%%%%%%%%%%%%%%%%%%%%%%%%%%%%%%%%%%%%%%%%%%%%%%%%%%%%%%%%%%%%%%
% Pervasive definitions
%%%%%%%%%%%%%%%%%%%%%%%%%%%%%%%%%%%%%%%%%%%%%%%%%%%%%%%%%%%%%%%%%%%%%%

% Notation
\newcommand{\ldot}{.\,}
\newcommand{\vd}{\null\mathrel{\vdash}}
\newcommand{\lb}{{[\![}}
\newcommand{\rb}{{]\!]}}
\newcommand{\op}[1]{\operatorname{#1}}

% Judgments
\newcommand{\jtrue}{\;\mbox{\it true}}
\newcommand{\jlax}{\;\mbox{\it lax}}

% seq calc
\newcommand{\proves}{\Rightarrow}
\newcommand{\up}{{\uparrow}}
\newcommand{\down}{{\downarrow}}

% Propositions
% \newcommand{\land}{\mathop{\wedge}}
\newcommand{\limplies}{\mathop{\supset}}
\newcommand{\ltrue}{\top}
% \newcommand{\lor}{\mathop{\vee}}
\newcommand{\lfalse}{\bot}
\newcommand{\lequiv}{\mathop{\equiv}}
\newcommand{\lbox}{\Box}
\newcommand{\ldia}{\Diamond}
% linear logic
\newcommand{\one}{\mathrm{1}}
\newcommand{\lolli}{\multimap}
\newcommand{\tensor}{\otimes}
\newcommand{\Tensor}{\bigotimes}
\newcommand{\bang}{!}
\newcommand{\mon}[1]{\{#1\}}
\newcommand{\with}{\&}
\newcommand{\aff}{@}


% Proofs
\newcommand{\ddd}{\raisebox{0.2em}[1.1em]{$\vdots$}}
\newcommand{\DD}{\mathcal{D}}
\newcommand{\EE}{\mathcal{E}}
\newcommand{\FF}{\mathcal{F}}
\newcommand{\GG}{\mathcal{G}}

% Judgments on proofs
\newcommand{\lred}{\mathrel{\raisebox{0.5em}{$\Longrightarrow_R$}}}
\newcommand{\lexp}{\mathrel{\raisebox{0.5em}{$\Longrightarrow_E$}}}
\newcommand{\lowerproof}[1]{\raisebox{-0.9em}{$#1$}}


\newcommand{\hastype}{\mathrel{:}}
\newcommand{\oftp}{\mathord{:}}

\newcommand{\lam}{\lambda}
\newcommand{\arrow}{\rightarrow}
% \newcommand{\fst}{\mbox{\bf fst}}
% \newcommand{\snd}{\mbox{\bf snd}}
\newcommand{\fst}{\pi_1}
\newcommand{\snd}{\pi_2}
\newcommand{\cross}{\mathop{\times}}

\newcommand{\unit}{\langle\,\rangle}

\newcommand{\inl}{\mbox{\bf inl}}
\newcommand{\inr}{\mbox{\bf inr}}
% \newcommand{\inl}{\iota_1}
% \newcommand{\inr}{\iota_2}
\newcommand{\ccase}{\mbox{\bf case}}
\newcommand{\oof}{\mathrel{\mbox{\bf of}}}
\newcommand{\bbar}{\mathrel{\mid}}
\newcommand{\plus}{\mathop{+}}

\newcommand{\abort}{\mbox{\bf abort}}
\newcommand{\zero}{\mbox{\bf 0}}

\newcommand{\pabort}{\mbox{\bf pabort}}
\newcommand{\pcase}{\mbox{\bf pcase}}

\newcommand{\pbox}{\mbox{\bf box}}
\newcommand{\letbox}{\mbox{\bf letbox}}

\newcommand{\pdia}{\mbox{\bf dia}}
\newcommand{\letdia}{\mbox{\bf letdia}}

\newcommand{\red}{\mathrel{\Longrightarrow}_R}
\renewcommand{\exp}{\mathrel{\Longrightarrow}_E}
% \newcommand{\jtype}{\mbox{\it type}}

% \newcommand{\nat}{\mathsf{nat}}
% \newcommand{\z}{\mathsf{0}}
% \newcommand{\s}{\mathsf{s}}

% \newcommand{\of}{{:}}
% \newcommand{\llet}{\mbox{\bf let}}
 \newcommand{\iin}{\mathrel{\mbox{\bf in}}}

% % Chapter 3: Proof Search
% \newcommand{\seq}{\Longrightarrow}
% \newcommand{\sq}{\longrightarrow}

% substitution
\newcommand{\psub}[1]{\langle\langle #1 \rangle\rangle}


%%% cmartens clf stuff

\newcommand{\LF}{LF} % for the LF signature

% operational judgments
\newcommand{\stepsto}{\rightsquigarrow}
%"proves in the focusing system"
\newcommand{\fproves}{\Longrightarrow}
\newcommand{\atm}[1]{\mathsf{#1}}
\newcommand{\stable}[1]{#1\mathsf{\ stable}}

% epsilons
\newcommand{\eps}{\epsilon}
\newcommand{\bnd}{\mathit{bnd}}
\newcommand{\nileps}{\langle\rangle}
% \newcommand{\gets}{\leftarrow}
% \newcommand{\gets}{=}

% phases
\newcommand{\type}{\mathsf{type}}
\newcommand{\phase}{\mathsf{phase}}
\renewcommand{\brack}[2]{[#1]^{#2}}
\newcommand{\qui}[1]{\mathtt{qui}\ #1}
\newcommand{\pha}[1]{\mathtt{phase}\ #1}

%modules
% \newcommand{\module}{\mathsf{module}}

% meta
\newcommand{\notmod}{\nvDash} % XXX?

% subordination
\newcommand{\subord}{\prec}
\newcommand{\nsubord}{\nprec}
\newcommand{\inits}{\mathsf{initials}}
\newcommand{\terms}{\mathsf{terminals}}
\newcommand{\interms}{\mathsf{intermediates}}



%%%%%%%%%%%%%%%%%%%%%%%%%%%%%%%%%%%%%%%%%%%%%%%%%%%%%%%%%%%%%%%%%%%%%%
% Pervasive definitions
%%%%%%%%%%%%%%%%%%%%%%%%%%%%%%%%%%%%%%%%%%%%%%%%%%%%%%%%%%%%%%%%%%%%%%

% Notation
\newcommand{\ldot}{.\,}
\newcommand{\vd}{\null\mathrel{\vdash}}
\newcommand{\lb}{{[\![}}
\newcommand{\rb}{{]\!]}}
\newcommand{\op}[1]{\operatorname{#1}}

% Judgments
\newcommand{\jtrue}{\;\mbox{\it true}}
\newcommand{\jlax}{\;\mbox{\it lax}}

% seq calc
\newcommand{\proves}{\Rightarrow}
\newcommand{\up}{{\uparrow}}
\newcommand{\down}{{\downarrow}}

% Propositions
% \newcommand{\land}{\mathop{\wedge}}
\newcommand{\limplies}{\mathop{\supset}}
\newcommand{\ltrue}{\top}
% \newcommand{\lor}{\mathop{\vee}}
\newcommand{\lfalse}{\bot}
\newcommand{\lequiv}{\mathop{\equiv}}
\newcommand{\lbox}{\Box}
\newcommand{\ldia}{\Diamond}
% linear logic
\newcommand{\one}{\mathrm{1}}
\newcommand{\lolli}{\multimap}
\newcommand{\tensor}{\otimes}
\newcommand{\Tensor}{\bigotimes}
\newcommand{\bang}{!}
\newcommand{\mon}[1]{\{#1\}}
\newcommand{\with}{\&}
\newcommand{\aff}{@}

\newcommand{\stepsto}{\rightsquigarrow}

% \usepackage{verbatim} %% verbatiminput, just while editing draft.

\usepackage{fancyvrb}
\fvset{fontsize=\tiny}
\RecustomVerbatimEnvironment{verbatim}{Verbatim}{}
\RecustomVerbatimCommand{\VerbatimInput}{VerbatimInput}{fontsize=\footnotesize}


\frenchspacing
\setlength{\pdfpagewidth}{8.5in}
\setlength{\pdfpageheight}{11in}
\pdfinfo{
% /Specifying Narrative Simulations in a Linear Logical Framework}
% /Specifying Generative Story Worlds in a Linear Logic Programming Language}
/Generative Story Worlds as Linear Logic Programs
/Author Chris Martens, Jo\~{a}o F. Ferreira, Anne-Gwenn Bosser, Marc
Cavazza}

\title{Generative Story Worlds as Linear Logic Programs}
\author{
Chris Martens \\ Carnegie Mellon University \\ \texttt{cmartens@cs.cmu.edu} \And
Jo\~{a}o F. Ferreira \\ Teesside University \\
\texttt{J.Ferreira@tees.ac.uk} \And
Anne-Gwenn Bosser \\ ENI Brest\\ Lab-STICC UMR6285 \\
\texttt{bosser@enib.fr} \And
Marc Cavazza \\ Teesside University \\ \texttt{M.O.Cavazza@tees.ac.uk}
}

% \author[1]{Chris Martens}
% \author[2]{Jo\~{a}o F. Ferreira}
% \author[3]{Anne-Gwenn Bosser}
% \author[2]{Marc Cavazza}
%\affil[1]{Carnegie Mellon University}
%\affil[2]{Teesside University}
%\affil[3]{University of Brest}


\setcounter{secnumdepth}{0}  
 \begin{document}

\maketitle

\begin{abstract}
\begin{quote}
Linear logic programming languages have been identified in prior
work as viable for specifying stories and analyzing their causal structure.
We investigate the use of such a language for specifying story {\em
worlds}, or settings where generalized narrative actions have uniform
effects (not specific to a particular set of characters or setting
elements), which may create emergent behavior through feedback loops.

We show a sizable example of a story world specified in the language Celf
and discuss its interpretation as a story-generating program, a simulation,
and an interactive narrative. Further, we show that the causal analysis
tools available by virtue of using a proof-theoretic language for
specification can assist the author in reasoning about the structure and
consequences of emergent stories.
\end{quote}
\end{abstract}

\section{Introduction}
% \label{sec:intro}

% % overarching vision: emergence & specification of processes

% Resource-oriented understanding of a story

% Planning, linear logic

% Cite prior work 

% Celf is a system implementing fwd-chaining linear logic programming

% fruitful for generation, simulation, and as a basis for adding
% interactivity.


Linear logic~\cite{girard87linear} (LL) has been proposed as a suitable conceptual
framework to specify and reason about interactive
narratives~\cite{Bosser10}, which has led to a variety of applications such
as narrative analysis~\cite{Bosser11}, narrative
generation~\cite{Martens13:LPNMR}, and authoring and validation
tools~\cite{DangHCS11}. At the same time, the success of
logically-motivated languages such as Inform
7\footnote{\url{http://www.inform7.com}} together with the documented
interest of Interactive Fiction (IF) writers  for emergent
narratives\footnote{See for instance Emily Short's article on emergence in
narrative interaction design at
\url{http://emshort.wordpress.com/2013/02/14/introducing-versu/}} suggests
that the IF community may welcome a programming language approach to the
authoring of rule-based, generative story worlds. 

Our work proposes the use of a logic programming language for specifying
emergent systems representing narrative worlds. We use a language based on
constructive LL, which supports action description, use of resources, and
changes imposed on the story world, and offers a basis for {\em analysis}
of storylines and causal relationships, all within the same theory of
proof.  This approach lets us have our cake (in the sense of designing
narratives with richly-interacting processes) and eat it, too (in the sense
of predicting and controlling the consequences of such processes).

\section{Related Work}

Planning systems have been widely adopted for the construction of
interactive narratives~\cite{Young99,Porteous10}, mostly because of their
support for the representation of causality. It is generally accepted that
LL is a strong candidate for such a representation~\cite{GirardILL87}, and
a proof in LL can be equated to a plan~\cite{Masseron93a,Masseron93b}.
Aside from surface differences, such as LL predicates having
meaningful multiplicity, the main benefit of using LL is that
planning actions, plan synthesis, and plans themselves can all be accounted
for uniformly under a minimal theory of inference and proof. 

% The proof-theoretic treatment we present opens the doors to extensibility
% (integration with orthogonal logical operators and features such as
% recursion~\cite{Cresswell2000deductive}), transformation techniques from
% compiler theory, and other work from decades of research in type theory and
% constructive proof theory. We see a logical basis for plans as something
% like another arm of the Curry-Howard correspondence (proofs as programs;
% surveyed recently in~\cite{Wadler14PropsAsTypes}), and expect in the long
% term to derive similar research benefits.

Standard logic programming approaches to narratives provide evidence for
the concise and readable nature of declarative
specification~\cite{Grasbon01,Lang99} but lack the native ability to model
state and causality afforded by LL. One logic programming engine for
managing generativity in games uses the Event Calculus (EC) to overcome this
shortcoming~\cite{Smith11}.

% For one account of linear logic's relationship to EC, see (cite Iliano?
% Alexiev? XXX).

LL has previously been explored for game analysis, as a front-end for Petri
nets~\cite{Colle2005}.  Further development demonstrated the value in using
a {\em constructive} formulation of LL to directly correlate the formal
notion of {\em story} with the notion of logical
proof~\cite{Bosser10,Bosser11}. Dang investigated LL for complete
exploration and validation of scenarios for educational
purposes~\cite{DangCA13}.  However, by relying on a backward-chaining proof
search interpretation, these works have favored authorial intent above
narrative emergence. Our contribution is to investigate LL for modeling
story scenarios which are primarily exploratory and generative, rather than
purely goal-driven, using forward-chaining proof search.
This duality between intent (backward chaining) and exploration
(forward chaining) has previously been used for combining deliberative and
reactive behaviour for agent modelling~\cite{Harland04}. Our language of
investigation, called Celf~\cite{schacknielsen08celf}, supports both
forward and backward chaining through the {\em focusing} theory of
proof search~\cite{chaudhuri10logical}.

\section{Contributions}

Expanding on our previous work~\cite{Martens13:LPNMR}, we demonstrate how
to encode more general story settings that are {\em parametric} over
arbitrary world states, settings, and characters. We demonstrate that these
rules create {\em nontrivial feedback loops} in terms of their causal
structure, allowing for long and perhaps unpredictable chains of events to
create pivotal events in stories.

We further extend our use of tools to this case, including those we get
for free from the language Celf as well as the extrinsic
graphical tool and query language. These tools illustrate the
use of proof term structure in exploring the consequences of rules which
may have been conceived in the haphazard, ad-hoc way of first draft
designs.

The takeaway point of this paper is that {\bf encoding story worlds in a
linear logic programming language allows for informative exploration of
their consequences.} For the remainder of the paper, we will support this claim by
describing an example and its semantics, demonstrating how
nondeterministic proof search can be used to {\em generate} stories, and
sketching our work in progress on tools for {\em analyzing} and {\em
interacting with} story models.

\section{System Overview}
% \label{sec:overview}

The basis of our framework is the use of LL as a language for specifying
actions. This setup is similar to the use of planning languages such as
STRIPS~\cite{Fikes1971STRIPS} in the sense that we model the world as a
collection of predicates and specify the available actions declaratively in
terms of those predicates.

A {\em world state} is an unordered collection of
predicates $\Delta$, and actions are written as state transitions $A \lolli
\{B\}$, signifying an action that replaces the state components
described by $A$ with those described by $B$.

The process of execution (or {\em simulation} or {\em generation}) of a
story, then, is to take a user-specified initial state $\Delta_0$, find
some rule $r$ in the specification $\Sigma$ that can apply, and apply it to get
$\Delta_1$. 
% We sometimes write this state transition as $\Delta_0 \stepsto^r \Delta_1$. 
% above notation never used.
This process is iterated until no further rules apply, at which
point we say we have reached {\em quiescence}.

When a rule $r : A \lolli \mon{B}$ is applied on state $\Delta, A$, the
next state is $\Delta, B$. The component $\Delta$ which is irrelevant to
the rule stays the same, solving the {\em frame problem} present in other
settings, such as event and situation calculi~\cite{hayes1971frame}. This
property arises from the inference rules defining $\lolli$ as the
proposition-level internalization of logical entailment.

We summarize the connectives and notation used below:

\begin{tabular}{ccc}
abstract syntax & concrete syntax & meaning\\
\hline
$A \lolli \mon{B}$    & \verb|A -o {B}|  & replacement \\
$A \tensor B$   & \verb|A * B|   & conjunction of resources\\
$\bang A$       & \verb|!A|      & persistent resource\\
\end{tabular}

Persistent resources may be regarded as permanent facts as in standard
propositional logics.

\section{Example: A Romantic Tragedy Story World}
% \label{sec:example}

The key to programming with LL is formulating one's problem in
terms of {\em resources}, i.e. the components of a story that may interact
and change. Then, the author describes {\em how} (through what actions)
those components change. In a bit more detail, the process for designing a
story world takes the following shape:

\begin{enumerate}
\item Identify the components of story state, such as physical location and
character relationships. Declare a predicate (type) for each of these; for
example, declare that \verb|anger C C'| is a well-formed state component
when $C$ and $C'$ are characters. We map predicates
in our example to their intended meanings in the table below.
% Table~\ref{table:predicates}.

\item Identify the {\em narrative actions}, i.e. ways that characters can
interact with each other or their environs to cause changes in the state.
For instance, a rule for flirting with a character other than one's lover
may cause anger in the lover directed at both flirters, which is
represented by the rule
\begin{verbatim}
do/flirt/conflict 
: eros Flirter Flirtee * eros Other Flirtee
  -o {eros Flirter Flirtee * eros Flirtee Flirter
      anger Other Flirter * anger Other Flirtee}.
\end{verbatim}
(The words beginning with capital letters, by convention, mark logic
variables, and are implicitly quantified at the beginning of the rule---the
names themselves are arbitrary.)
%Effectively, such rules can be read as a directive to the system, ``replace
%the state described to the left of \verb|-o| with the state described to
%the right of it.''
\end{enumerate}

A complete story world specification is simply a collection of these
predicate and rule declarations.  The author may then join it with its
complementary half, a specification of setting elements, characters, and
initial states, to generate stories or analyze the system we have
described.  The remainder of this section carries out an example, a case
study of a Shakespeare-inspired romantic tragedy world, following
the aforementioned workflow.\footnote{The complete, runnable code for this
example can be found at
\url{https://github.com/chrisamaphone/interactive-lp/blob/master/examples/tragedy.clf}.}

The state we model includes sentiments between characters (several forms of
love and hate), physical locations and basic movement, possession of key
objects (such as weapons), relationship states (marriage and being single),
desires for objects, and solitary emotions (depression).

% \begin{table}[ht]
%\centering
\begin{tabular}{c|c}
Predicate & Meaning\\
\hline
\verb|at C L|       & $C$ is (alive) in location $L$\\
\verb|has C O|      & $C$ possesses an object $O$\\
\verb|neutral C C'| & $C$ feels neutrally toward $C'$\\
\verb|philia C C'|  & $C$ feels affection toward $C'$\\
\verb|anger C C'|   & $C$ feels anger toward $C'$\\
\verb|eros C C' |   & $C$ feels attraction toward $C'$\\
\verb|unmarried C|  & $C$ is unmarried\\
\verb|married C C'| & $C$ is married to $C'$\\
\verb|depressed C|  & $C$ is depressed\\
\verb|suicidal C|   & $C$ is suicidal\\
\verb|!dead C|      & $C$ is dead\\
\verb|!murdered C C'| & $C$ murdered $C'$\\
\verb|!actor C|     & $C$ is a character in the story
\end{tabular}
% \label{table:predicates}
% \end{table}

These predicates do not specify a particular cast of characters or setting,
but they could be said to pertain to a particular {\em genre} of story, in
this case romance and tragedy. The delineation of the story world into
predicates is an act of careful human design, often iterated with the
generation of stories, and the choices made here make all the difference in
terms of which actions are possible to write and therefore what shapes the
story can take. For instance, the choice to include two kinds of love,
\verb|eros| and \verb|philia|, means that we can codify social rules about
sexual relationships between characters. Similarly, the choice {\em not} to
include a predicate for a character's gender renders it impossible to
enforce heteronormative relationships. We have to make a choice about
whether we want to model a realistic description of human behavior (which
often contradicts social norms) or enforce social norms by constraining
actions with our formal description. 

A selection of rules for our chosen genre is given below, starting with the
rules for basic social interaction:\footnote{The location predicates
\texttt{at C L} in these rules signify not just location, but also that the
character in question is alive and available in the story. The \texttt{at}
atom is consumed when a character dies.}

\begin{verbatim}
do/formOpinion/like
: at C L * at C' L *
  neutral C C'
  -o {at C L * at C' L * philia C C'}.

do/formOpinion/dislike
: at C L * at C' L *
  neutral C C'
  -o {at C L * at C' L * anger C C'}.

do/compliment/private
: at C L * at C' L * philia C C' 
  -o {at C L * at C' L * philia C C' * philia C' C}.

do/compliment/witnessed
: at C L * at C' L * at Witness L * philia C C' *
  anger Witness C' 
  -o {at C L * at C' L * at Witness L * philia C C' * 
      anger Witness C' * philia C' C * anger Witness C}.

do/insult/private
: at C L * at C' L * anger C C' 
  -o {at C L * at C' L * anger C C' * anger C' C * 
      depressed C'}.

do/insult/witnessed
: at C L * at C' L * at Witness L * 
  anger C C' * philia Witness C' 
  -o {at C L * at C' L * at Witness L * 
      anger C C' * philia Witness C' * anger C' C * 
      depressed C' * anger Witness C}.

mixed_feelings 
: at C L * anger C C' * philia C C' -o {at C L * neutral C C'}.
\end{verbatim}
Note that there are two ``versions'' each of the actions for insulting and
complimenting, one that happens ``in private'' and another that affects a
witness in the same location. This encoding signals a weakness: it is
unnatural to represent a {\em broadcast} of an action affecting every
character who might be in range, since LL primarily codifies {\em local}
state changes. On the other hand, this mechanism's nondeterminism could be
argued to reflect the chance involved in whether an action goes noticed.

Next we codify the rules for romantic interaction, which include
tranformations between eros and philia as well as flirting, marriage, and
divorce:
\begin{verbatim}
do/fallInLove
: at C L * at C' L' *
  eros C C'
  -o {at C L * at C' L' * eros C C' * philia C C'}.

do/eroticize
: at C L * at C' L' *
  philia C C' * philia C C' * philia C C' * philia C C'
  -o {at C L * at C' L' * philia C C' * eros C C'}.

do/flirt/ok
: at C L * at C' L * eros C C' * unmarried C * unmarried C'
               -o {eros C C' * eros C' C * 
                    unmarried C * unmarried C' *
                    at C L * at C' L}.
do/flirt/discreet
: at C L * at C' L * eros C C'
  -o {eros C C' * eros C' C * at C L * at C' L}.

do/flirt/conflict 
: at C L * at C' L * at C'' L *
  eros C C' * eros C'' C
                -o {eros C C' * eros C' C * eros C'' C 
                    * anger C'' C' * anger C'' C
                    * at C L * at C' L * at C'' L}.
do/marry 
: at C L * at C' L * 
  eros C C' * philia C C' * 
  eros C' C * philia C' C * 
  unmarried C * unmarried C'
  -o {married C C' * married C' C * at C L * at C' L *
      eros C C' * eros C' C * philia C C' * philia C' C }.

do/divorce
: at C L * at C' L' *
  married C C' * married C' C * anger C C' * anger C C'
  -o {anger C C' * anger C' C * unmarried C * unmarried C'
      * at C L * at C' L'}.

do/widow
: married C C' * at C L * dead C'
  -o {unmarried C * at C L}.
\end{verbatim}

These rules include the generation of sentiments from a netural stance,
transformations between the two kinds of love \verb|philia| and \verb|eros|,
and flirting, which strengthens both kinds of love, but causes anger if
witnessed by another paramour. We also include rules that modify the
marriages of characters.

Next we supply rules governing death and violence:
\begin{verbatim}
do/murder 
: anger C C' * anger C C' * anger C C' * anger C C' *
  at C L * at C' L  * has C weapon
  -o {at C L * !dead C' * !murdered C C' * has C weapon}.

do/becomeSuicidal
: at C L *
  depressed C * depressed C * depressed C * depressed C
  -o {at C L * suicidal C * wants C weapon}.

do/comfort
: at C L * at C' L *
  suicidal C' * philia C C' * philia C' C
  -o {at C L * at C' L * 
      philia C C' * philia C' C * philia C' C}.

do/suicide
: at C L * suicidal C * has C weapon -o {!dead C}. 

do/mourn 
: at C L * philia C C' * dead C' 
  -o {at C L * depressed C * depressed C}.

do/thinkVengefully
: at C L * at Killer L' * 
  philia C Dead * murdered Killer Dead
  -o {at C L * at Killer L' * philia C Dead *
      anger C Killer * anger C Killer}.
\end{verbatim}

The violence module introduces several potential feedback loops between
murder and vengeance, suicide, mourning, and depression.

Finally, we have a few actions that can affect possession:
\begin{verbatim}
do/give
: at C L * at C' L * has C O * wants C' O * philia C C'

do/steal
: at C L * at C' L * has C O * wants C' O
  -o {at C L * at C' L * has C' O * anger C C'}.

do/loot
: at C L * dead C' * has C' O * wants C O
  -o {at C L * has C O}.
\end{verbatim}

Given this set of rules, we note that the story world is multi-agent in
nature---the interactor with such a story doesn't obviously ``play'' one
particular character, and the rules aren't defined as ``behaviors''
attached to a given agent. They portray the interiority of all characters
at once, allowing them to be referenced and changed in combination.

\subsection{Initial State}

After describing the general rules of our Shakespearean tragedy
story world, which are parameterized over characters and locations, we can
fill in specific elements, such as the characters and setting of Romeo and
Juliet, to have a complete and runnable specification.

First we can describe the {\em persistent} (unchanging, i.e. not linear)
facts about the story, in this case the world map and the cast of
characters:

\begin{verbatim}
mon/town : accessible mon_house town.
town/mon : accessible town mon_house.
cap/town : accessible cap_house town.
town/cap : accessible town cap_house.

a-romeo : actor romeo.
a-juliet : actor juliet.
a-montague : actor montague.
a-capulet : actor capulet.
a-mercutio : actor mercutio.
a-nurse : actor nurse.
a-tybalt : actor tybalt.
a-apothecary : actor apothecary.
\end{verbatim}

Next, we need to designate the {\em initial} state of all the linear
predicates. It could look something like this:

\begin{verbatim}
story_start :
init -o { at romeo town * at montague mon_house * at capulet cap_house
  * at mercutio town * at nurse cap_house * at juliet town
  * at tybalt town * at apothecary town

  * has tybalt weapon * has romeo weapon * has apothecary weapon

  * unmarried romeo * unmarried juliet
  * unmarried nurse * unmarried mercutio * unmarried tybalt
  * unmarried apothecary

  * anger montague capulet * anger capulet montague
  * anger tybalt romeo * anger capulet romeo * anger montague tybalt

  * philia mercutio romeo * philia romeo mercutio
  * philia montague romeo * philia capulet juliet
  * philia juliet nurse * philia nurse juliet

  * neutral nurse romeo
  * neutral mercutio juliet * neutral juliet mercutio
  * neutral apothecary nurse * neutral nurse apothecary}.
\end{verbatim}

The first two groups of atoms describe the story world locations where
the characters begin and their possessions. The next group represents which
characters are unmarried at the start of the story. The next two groups
represent existing relationships (sentiments)  among characters, and
finally the last group represents which characters haven't met each other
yet (and so feel neutrally towards each other).

\subsection{Serendipity}

While general-purpose, generative story rules serve to create an emergent
notion of story, one might argue that in many stories, {\em coincidence} or
{\em serendipity} plays a large role in the story being interesting. We
have a way to codify the idea of serendipitous events as well: it is that
{\em rules} of the form $A \lolli \mon{B}$ can be treated on the same level as
other propositions, and thus included in initial states.

For example, we encode Romeo and Juliet's ``love at first sight'' as a
single rule added to the consequent of \verb|story_start|:

\begin{verbatim}
story_start :
init -o {
  ...
  {Forall L. at romeo L * at juliet L
      -o {eros romeo juliet * at romeo L * at juliet L}}
  ...}.
\end{verbatim}

This single-use rule dictates that whenever Romeo and Juliet are in the
same place, Romeo may gain eros for Juliet.  The construct 
\verb|Forall L ...| allows the rule to be parameterized over the shared
location where they meet. (This use of quantification differs from the
implicit quantification happening on the outside of every rule; here, we
want to bind the location {\em locally} rather than over the entire
\verb|story_start| rule.)

\subsection{Final States}

We could at this point call the example complete and ask the system to
begin executing the rules from the given initial state. But before doing
so, we might be interested in some high-level structural questions, such
as: do stories in this specification {\em end}? If so, how do they end?

Answering this question requires thinking about the rules {\em
operationally} rather than as static descriptions of possible actions. The
operational semantics of the program is based on the formal description
given in the section System Overview wherein a {\em step} is an evolution
of a context $\Delta, A$ to a context $\Delta, B$ along a rule $A \lolli
\mon{B}$.\footnote{This is a simplification of the story---the true
transition semantics are given by rules that do not require
the rule's precondition to be already formed in the context, but may
perform backward search to find it.} The simulation terminates when no more
steps can be made, i.e.  when no more rules apply to the current context.

We make the following observation: all of the character actions in our
specification require at least one character to be alive (represented by
the \verb|at| predicate). Most of the rules preserve location/aliveness of
the characters, but the actions corresponding to character deaths (murder
and suicide) do not. Therefore, the story will terminate when all
characters have died.

At several points in the story, multiple rules will apply. If we consider a
{\em fair} operational semantics, i.e. one where when multiple rules apply,
each has some positive chance of being chosen, then {\em eventually} the
termination condition will be reached. How long the story goes on before it
ends is a function of the more specific probabilities a rule is selected
with in the story engine -- which property is not defined by the language
semantics, but is observable of the implementation.

For this example, we decided to make termination conditions that would be
met more frequently for the sake of shorter stories, more dense with
interesting behavior. This also allows us to codify a set of ``story
endings.'' To do this, we create new atoms \verb|final| and
\verb|nonfinal|, add \verb|nonfinal| to the initial state,
and write a few rules that lead from desired end conditions to
the \verb|final| atom, consuming \verb|nonfinal|, e.g.:

\begin{verbatim}
ending_happy
: nonfinal *
  actor C * actor C' *
  at C L * at C' L * married C C'
  -o {final}.

ending_vengeance
: nonfinal *
  actor C1 * actor C2 * actor C3 *
  killed C1 C2 * philia C3 C2 * killed C3 C1
    -o {final}.
\end{verbatim}

\section{Proofs as Stories}
% \label{sec:traces}

To reiterate, the takeaway point of this paper is that {\em encoding
story worlds in a linear logic programming language allows for informative
exploration of their consequences}.  Two of these exploration techniques
are {\em random story generation} and {\em structural analysis}, and what
enables those techniques to fall naturally out of our encoding is the fact
that there is a direct and formal correspondence between {\em stories} and
{\em proofs}. 

Once we write our specification as a collection of logical
formulas, we can initiate a {\em query} such as

\begin{verbatim}
?- init -o {final}.
\end{verbatim}

This query asks whether there is a {\em proof} from the state described by
\verb|init| to the atom \verb|final|. Initially, proof search is {\em
goal-directed} or backward-chaining: it adds \verb|init| to the current
state and considers how to prove \verb|{final}|. Before now, we have been
treating the curly braces \verb|{-}| as meaningless, but they mean
something in terms of proof search: it must now switch to a {\em
forward-chaining} or {\em generative} mode, running inference forward from
the \verb|init| atom. Only once the entire story has terminated will it
look for \verb|final|, at which point search will succeed if it finds it.

But the point of executing the query isn't really to find out whether the
initial state leads to a valid conclusion. What we are interested in is the
{\em trace} generated by execution---the witness to the validity of the
proposition, i.e. the proof!

Interpreting propositions as types, we assign a proof term to the
implication $A \lolli B$ as a $\lambda$-term, or function, from terms of
type $A$ to terms of type $B$. Within the body $M$ of the function
$\lambda{x{:}A}.M$, the story term can make use of the variable $x$, e.g.
by projecting out its components and applying rules in the story signature
to them.

Proofs of monadic goals, such as, in our example, \verb|{final}|, are lists
of let-bindings that capture the trace of actions, e.g.:

%See Figure~\ref{fig:letbind}
%for a segment of this let-binding structure generated by Celf on the
%example given.

% \begin{figure*}
% \caption{A fragment of Celf trace/structured plot.
% \label{fig:letbind}}
% \centering
% \begin{verbatim}
% ...
% let {[X73, [X74, [X75, [X76, X77]]]]} 
%   = do/insult/private [a-tybalt, [a-romeo, [X68, [X66, X72]]]] in 
% let {[X78, [X79, [X80, [X81, [X82, [X83, X84]]]]]]} 
%   = do/compliment/witnessed [a-mercutio, [a-romeo, [a-tybalt, [X67, [X74, [X73, [X57, X56]]]]]]] in 
% let {[X85, [X86, X87]]} = do/becomeSuicidal [a-romeo, [X79, [X41, [X59, [X52, X77]]]]] in 
% let {[X88, [X89, [X90, [X91, X92]]]]} 
%   = do/comfort [a-mercutio, [a-romeo, [X78, [X85, [X86, [X81, X83]]]]]] in 
% let {[X94, [X95, [X96, [X97, [X98, [X99, X100]]]]]]} 
%   = do/compliment/witnessed [a-romeo, [a-mercutio, [a-tybalt, [X89, [X88, [X80, [X92, X84]]]]]]] in 
% let {[X101, [!X102, [!X103, X104]]]} 
%   = do/murder [a-romeo, [a-tybalt, [X58, [X40, [X76, [X51, [X94, [X96, X27]]]]]]]] in 
% let {[X105, [X106, [X107, X108]]]} 
%   = do/compliment/private [a-nurse, [a-juliet, [X46, [X47, X30]]]] in 
% let {[X109, [X110, [X111, X112]]]} 
%   = do/compliment/private [a-juliet, [a-nurse, [X106, [X105, X108]]]] in 
% let {[X113, X114]} 
%   = do/loot [a-romeo, [a-tybalt, [X101, [X102, [X26, X87]]]]] in
% ...
% \end{verbatim}
% % \includegraphics[width=0.5\textwidth]{let-binding-annotated}
% \end{figure*}

\begin{verbatim}
...
let {[X73, [X74, [X75, [X76, X77]]]]} 
  = do/insult/private [a-tybalt, [a-romeo, [X68, [X66, X72]]]] in 
let {[X85, [X86, X87]]} 
  = do/becomeSuicidal [a-romeo, [X79, [X41, [X59, [X52, X77]]]]] in 
let {[X88, [X89, [X90, [X91, X92]]]]} 
  = do/comfort [a-mercutio, [a-romeo, 
      [X78, [X85, [X86, [X81, X83]]]]]] in 
let {[X101, [!X102, [!X103, X104]]]} 
  = do/murder 
    [a-romeo, [a-tybalt, 
      [X58, [X40, [X76, [X51, [X94, [X96, X27]]]]]]]] in 
let {[X105, [X106, [X107, X108]]]} 
  = do/compliment/private 
    [a-nurse, [a-juliet, [X46, [X47, X30]]]] in 
let {[X109, [X110, [X111, X112]]]} 
  = do/compliment/private 
    [a-juliet, [a-nurse, [X106, [X105, X108]]]] in 
let {[X113, X114]} 
  = do/loot [a-romeo, [a-tybalt, [X101, [X102, [X26, X87]]]]] in
...
\end{verbatim}

This fragment of trace shows how certain scenes, such as one wherein Tybalt
drives Romeo to murder with Mercutio's support, are interleaved with
independent scenes, such as a loving conversation between the Nurse and
Juliet.

Within the let-binding portion of the trace, we see a record of the story
rules selected, which can be seen as a linear progression of events. But
additionally, each binding 
\verb|let {[X1, ..., Xn]} = rule [Y1, ..., Ym]| represents
the call of \verb|rule| on previously-generated resources \verb|Y1 ... Ym|
representing its antecedent, generating {\em new} resources 
\verb|X1 ...  Xn| representing its consequent. This allows us to analyze
the proof term for dependency structure, specifically revealing which
events are {\em in}depedenent and can be thought of as concurrent
storylines. 

Here is a graphical depiction of the let-binding for Mercutio comforting
Romeo in the above trace fragment:

\includegraphics[width=0.5\textwidth]{let-binding-annotated}

% As a specific example, consider the line of the figure wherein
% Mercutio comforts Romeo. The following image illustrates 
% the proof term syntax in terms of input and output resources to a rule:
% XXX point out figure and image

% \begin{verbatim}
% X78 : at mercutio L
% X85 : at romeo L
% X86 : suicidal romeo
% X81 : philia mercutio romeo
% X83 : philia romeo mercutio
% \end{verbatim}
% 
% \begin{verbatim}
% X88 : at mercutio L
% X89 : at romeo L
% X90 : philia mercutio romeo
% X91 : philia romeo mercutio 
% X92 : philia romeo mercutio
% \end{verbatim}

The variables at the top of the image represent the inputs/antecedents to
the rule, including the persistent witnesses of Romeo and Mercutio's
actorhood, which arise from the signature, as well as those generated by
prior let-bindings. The variables in the bottom are freshly generated by
the rule, representing its outputs/consequents.

Although the let-bindings appear presented linearly, because they encode
data dependency information, independent forks can actually be extricated
from one another through a notion of {\em concurrent equality} derived from
the syntactic structure, i.e.
\verb|let x1 = M1 in let x2 = M2 in M| may be considered the same trace as
\verb| let x2 = M2 in let x1 = M2 in M|
iff the inputs of \verb|M2| are separate from the outputs of \verb|M1|.

It is important to emphasize the critical use of Celf's basis in
constructive logic, which enables the identification of program execution
with a structured witness to examine, complete with dependency structure.

% Some discussion of what happens if we include, exclude, or vary the rules -
% variations on the specification that work like ``knobs'' in the story
% output

\subsection{Graphical Analysis}

For the sake of making the dependency structure more visually explicit, we
have been developing a tool that automatically translates generated stories
into causal diagrams in the form of directed graphs. Nodes of the graph are
actions in the story, and edges can be viewed as causal relationships
between story events involving those actions.

This tool takes a proof term and the story specification as inputs, and it
generates the extrapolation of the let-binding illustration
given in the previous section to the whole trace: we wind up with a causal
network of action nodes representing the story's nonlinear progression of
events, e.g.:

\includegraphics[width=0.5\textwidth]{graph6}

% These graphs have edges from \verb|init| to each action that requires
% resources present in the initial environment.

The tool also has an interface for {\em queries} on sets of traces. E.g.
the query \verb|exists ending_1|
would tell us the set of stories with \verb|ending_1| (a marriage and a
death).
This tool allows us to test our specifications for how closely they match
our authorial intent. For instance, if we want to find out if any stories
contain unfulfilled vengeance, we might issue the query

%---e.g. perhaps we decide that we would not like to end
%the story if there is a storyline in which a character became suicidal but
%has not yet been comforted or performed suicide. We can observe the
%violation of this causal circumstance in the graph, make changes to the
%specification, and issue the query

\begin{verbatim}
exists do/thinkVengefully && ~link do/thinkVengefully do/murder
\end{verbatim}

(read as: there exists a ``think vengefully'' action, but there is no link
between ``think vengefully'' and ``murder'')
which might tell us that no stories satisfy the predicate. If we intend for
vengeance to occur more or less frequently, we may tune the parameters of
the rules (e.g. how many \verb|anger| atoms it generates) and run the query
again.

% \begin{figure*}
% \caption{A graphical rendering of a Celf trace.
% \label{fig:graph}}
% \centering
% \includegraphics[width=0.5\textwidth]{graph6}
% \end{figure*}

%%% This part didn't happen.
% However, unlike Celf traces, which distinguish between multiple resources
% of the same type (e.g. an \verb|anger| atom arising from a direct insult
% or from an unfaithful lover), the tool uses a notion of proof
% irrelevance~\cite{LeyWild07tr} to group the possible generators of a
% resource in a disjunction, called an ``or-node.'' Multiple in-edges to a
% (regular) node, then, unambiguously represent that the {\em conjunction}
% of the resources is needed for that action.

% XXX give a small illustration

% In figure XXX, we show a full causal diagram for one trace of the example
% specification.  (XXX some commentary?)


\section{Ongoing Work}
% \label{sec:ongoing}

Suppose we wanted to treat the same logic program not as a random story
generator but as an {\em interactive} story.  At a first cut, a player
could interact with proof search by selecting a rule whenever proof search
would otherwise select one randomly between several that apply. But in the
interest of creating a more exploratory feel, rather than present a finite
set of available story branches, we wish to follow the interactive fiction
tradition of allowing a general {\em grammar} of actions. The action
language for our example might correspond loosely to the available rules,
e.g.

\begin{verbatim}
insult(C, C')  compliment(C, C')  flirt(C, C')  marry(C, C') 
murder(C, C')  comfort(C, C')  suicide(C)  divorce(C, C') 
\end{verbatim}

Parameterizing actions over all of the characters they concern supposes an
omnicient point of view where the interactor may control all characters. If
we suppose a single-character point of view, then the first parameter
\verb|C| in the actions above becomes implicit.
After giving this language of actions, we can modify the original program
so that the rules concerning, e.g., flirtation, take an extra precondition
that an atom \verb|do(flirt(C, C'))| appears. This atom should be linear so
that the rule only fires once.

We are working on an extension to the core language underlying Celf that
allows the introduction of action atoms by the player once the
program has reached quiescence, which we believe soundly captures the idea
of interpreter and interactor taking turns. We formalize this idea of
turn-taking as a language construct called {\em
phases}~\cite{Martens13:thesis-proposal}, which turn out to be generally
useful for other engineering concerns, such as structuring a program into
independent (module-like) components. Further, phases enable us to
describe the {\em broadcast} behavior we were missing earlier: broadcast
can be its own phase that iterates a rule to quiescence over all occupants
of a room.
After the introduction of phases, we believe the language will be suitable
as the core of a system for authoring large, expressive works of
interactive fiction and exploratory interactive simulation (``sandbox''
games). In the context of a larger project, of course, we would also need
tools for parsing and generating natural language.

We are also developing a more complete suite of analysis capabilities, akin
to model checking, that would allow authors to predict and control the
range of narrative possibilities, perhaps choosing to restrict them to a
more linear causality tree (in the case of a rigid narrative) or relax the
constraints (in the case of a sandbox world).  We are designing a
verification-like system overlaying the language wherein the author may, on
a per-phase basis, write down properties (invariants, preconditions, and
postconditions) to the logic program, which may then be automatically
checked. An application of this functionality might include ensuring as an
invariant that, whenever there is a locked door, a key is reachable from
the player's location.

\section{Summary of Contributions}
% \label{sec:summary}

We see this work as making the following contributions to
narrative technologies:
\begin{itemize}
\item A programming language-based approach to authoring interactive
stories, founded on a theory which yields an interpretation of stories as
logical proofs.
\item Demonstration of how said interpretation enables the {\em generation},
{\em analysis}, and {\em interactive interpretation} of stories from
story worlds. 
\item Atop prior work in formalizing stories in linear logic,
we show that this approach can be used not only for {\em validating}
goal-driven story interactions but also for sculpting exploratory,
reactive experiences.
\end{itemize}

Ongoing work should allow for combining exploratory and verificational
approaches to authorship, laying the theoretical groundwork for a
language for generative interactive fiction.

\section{Acklowledgments}

The authors thank the anonymous reviewers for their thorough and
encouraging feedback. The first author thanks Frank Pfenning, Martin van Velsen,
and Adam Smith for conversations that helped contextualize our work.

% As evidenced by methods preferred by practicing authors including
% logically-motivated languages such as Inform
% 7~\footnote{\url{http://www.inform7.com}}, and continued interest in
% interactive social simulation systems such as
% Versu~\footnote{http://emshort.wordpress.com/2013/02/14/introducing-versu/}
% (despite its recent cancelation), we see that the interactive storytelling
% community is ready for an approach from a programming language point of
% view for writing rule-based, generative story-worlds. We view this work as
% a step toward laying the theoretical groundwork on which an ideal language,
% in the sense of authoring concepts matching closely with language
% constructs without sacrificing any generality of the language, could be
% based.

\bibliographystyle{aaai}
\bibliography{main}

\end{document}
